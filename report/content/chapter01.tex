\chapter{Tổng quan về trò chơi Caro}
\setlength{\epigraphwidth}{0.5\textwidth}

\section{Tên gọi}

Cờ ca-rô, trong tiếng Việt \emph{ca-rô} (hay \textit{sọc ca-rô}) được phiên âm từ \emph{carreau} trong tiếng Pháp là các các ô vuông được sắp đều trên một mặt bề mặt. Cờ ca-rô trong tiếng Triều Tiên là omok\begin{CJK*}{UTF8}{mj} 오목 \end{CJK*}, tiếng Trung là\begin{CJK*}{UTF8}{gbsn} 五子棋 \end{CJK*} và trong tiếng Nhật là \begin{CJK*}{UTF8}{min}五目並べ\end{CJK*} (gomokunarabe). Tiếng Anh sử dụng lại tiếng Nhật, gọi là Gomoku. Gomoku đã tồn tại ở Nhật Bản từ trước thời Duy Tân Minh Trị (1868). ``Go'' trong ``Gomokunarabe'' tiếng Nhật nghĩa là \emph{năm}, ``moku'' là nghĩa là \emph{quân cờ} và ``narabe'' có nghĩa là \emph{được xếp hàng}.

\section{Ý tưởng}

Mỗi người chơi sẽ được dùng viên đá cùng màu (hoặc có thể dùng một ký hiệu nếu chơi trên các môi trường khác). Các người chơi thay phiên nhau đặt một viên đá lên trên một bảng trống có kẻ các ô vuông. Người chiến thắng là người chơi đầu tiên tạo thành một chuỗi liên tục gồm năm viên đá theo chiều ngang, chiều dọc hoặc đường chéo. Việc đặt sao cho một chuỗi có nhiều hơn năm viên đá cùng loại được tạo ra sẽ không được xem là chiến thắng. Có nhiều cách chơi quy định rằng mỗi chuỗi 5 cùng màu được xem là không hợp lệ nếu hai đầu chuỗi được đặt bởi hai viên đá khác. Trong bài tiểu luận tác giả sẽ không sử dụng luật này.